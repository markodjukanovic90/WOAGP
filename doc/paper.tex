\documentclass[runningheads,a4paper]{llncs}
% vim: tw=0 wm=0

\setcounter{tocdepth}{3}
\usepackage{amssymb}
\usepackage{amsmath}
%\usepackage{amsthm}
\usepackage{bbm}
\usepackage{environ}
\usepackage{multirow}
\usepackage{longtable}
\usepackage{comment}
\usepackage{placeins}
\usepackage{mathtools}
%\usepackage{algorithmic}
\usepackage{enumitem}
\usepackage[utf8]{inputenc}
%\usepackage{enumite}
%\usepackage{cleveref}
%\usepackage{parskip}
\usepackage{algpseudocode}
\usepackage{algorithm}
\usepackage{array}
\usepackage[pdfencoding=auto,psdextra]{hyperref}
\usepackage{booktabs}
\usepackage{bookmark}% faster updated bookmarks
\usepackage{hypcap} % fix the links
\evensidemargin\oddsidemargin
\usepackage{graphicx}
\pagestyle{plain}
\usepackage{xcolor}
\newcommand\ToDo[1]{\textcolor{red}{#1}}
%\bibliographystyle{plainnat}
\usepackage{siunitx}
\usepackage{color}

\usepackage[draft,nomargin,inline]{fixme}
\fxsetface{inline}{\itshape}
\fxsetface{env}{\itshape}
%\fxuselayouts{margin}
%\fxuselayouts{inline}
\fxusetheme{color}

\usepackage{url}
\urldef{\mailsa}\path|{djukanovic, raidl}@ac.tuwien.ac.at,|
\urldef{\mailsb}\path|christian.blum@iiia.csic.es|
\newcommand{\keywords}[1]{\par\aDSvspace\baselineskip
\noindent\keywordname\enspace\ignorespaces#1}

\usepackage{tikz}
\usetikzlibrary{positioning}
\definecolor{canaryyellow}{rgb}{1.0, 0.94, 0.0}
\definecolor{brightgreen}{rgb}{0.4, 1.0, 0.0}
\definecolor{jazzberryjam}{rgb}{0.65, 0.04, 0.37}

%defining of command

\newcommand\floor[1]{\lfloor#1\rfloor}
\newcommand\ceil[1]{\lceil#1\rceil}
\newcommand\str[1]{\texttt{#1}}
\newcommand\pL[1][]{\ensuremath{p^{\mathrm{L}#1}}}
\newcommand\pR[1][]{\ensuremath{p^{\mathrm{R}#1}}}
\newcommand\qL{\ensuremath{q^\mathrm{L}}}
\newcommand\qR{\ensuremath{q^\mathrm{R}}}
\newcommand\pLH{\ensuremath{\hat{p}^\mathrm{L}}}
\newcommand\pRH{\ensuremath{\hat{p}^\mathrm{R}}}
\newcommand{\Vext}{\ensuremath{V_\mathrm{{ext}}}}
\newcommand\UB{\ensuremath{\mathrm{UB}}}
\newcommand\Sigmand{\ensuremath{\Sigma^\mathrm{nd}}}
\renewcommand{\labelenumii}{\theenumii}
\renewcommand{\theenumii}{\theenumi.\arabic{enumii}.}
\setlength{\leftmarginii}{1.8ex}
\raggedbottom
\algnewcommand\algorithmicforeach{\textbf{for each}}
\algdef{S}[FOR]{ForEach}[1]{\algorithmicforeach\ #1\ \algorithmicdo}

% scaling factor for tables
\newcommand\tabscale{0.8}

\begin{document}

    %\setlength{\parindent}{0pt}  % disallow indentations
    %\numberwithin{table}{1}
    \mainmatter  % start of an individual contribution

    % first the title is needed
    \title{Greedy Heuristics for Solving the Weighted Orthogonal Art Gallery Problem}

    %
    \author{--}
    %

    \institute{%$^1$Institute of Logic and Computation, TU Wien,
    %Vienna, Austria,\\
    %	       $^2$ Artificial Intelligence Research Institute (IIIA-CSIC),\\ \normalsize Campus UAB, Bellaterra, Spain \\
    %\institute{Springer-Verlag, Computer Science Editorial,\\
    %Tiergartenstr. 17, 69121 Heidelberg, Germany\\
    %\mailsa\\
    %\mailsb\\
    %\mailsc\\
    %\url{http://www.springer.com/lncs}
    }

    \maketitle


    \section{Introduction}\label{sec:introduction}
     \emph{The Orthogonal Art Gallery Problem} (OAGP) is one of many variants of Art Gallery Problem (AGP). AGP asks for a set of points $G$ of minimal cardinality on some polygon $P$ such that for each point $y \in P$ there is $x \in G$ such that $xy \subset P$.  Set $P$ is called guard set of $P$, and the points from $G$ as guards. The orthogonal AGP consider on arbitrary polygon but whose angles are $90^{\deg}$ and $270^{\deg}$. This problem is stated by Victor  Klee in 1973.~\cite{o1987art}. The problem is motivated from installing the cameras in a building (or gallery) such that a maximal surface is covered (possibly all areas are covered). Orthogonality comes that most of the walls are orthogonal to some other in a building. A variant of the OAGP for which we are interested in this studies allows only that guards are positioned at the edges of polygon $P$. It is known to be NP--hard~\cite{devadoss2011discrete}.
     
     Since then, many approaches have been proposed to solve the problem.
     \fxnote{TODO: work on literature approaches}
     \section{Preliminaries}
     
     \section{Algorithmic Approaches}
     
     \section{Computational Results}
     
     \section{Conclusions and Future Work}
     
     
  
    \bibliographystyle{abbrv}
    \bibliography{bib}


\end{document}
