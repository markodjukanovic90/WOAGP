\documentclass[runningheads,a4paper]{elsarticle}
% vim: tw=0 wm=0

\setcounter{tocdepth}{3}
\usepackage{amssymb}
\usepackage{amsmath}
%\usepackage{amsthm}
\usepackage{bbm}
\usepackage{environ}
\usepackage{multirow}
\usepackage{longtable}
\usepackage{comment}
\usepackage{placeins}
\usepackage{mathtools}
%\usepackage{algorithmic}
\usepackage{enumitem}
\usepackage[utf8]{inputenc}
%\usepackage{enumite}
%\usepackage{cleveref}
%\usepackage{parskip}
\usepackage{algpseudocode}
\usepackage{algorithm}
\usepackage{array}
\usepackage[pdfencoding=auto,psdextra]{hyperref}
\usepackage{booktabs}
\usepackage{bookmark}% faster updated bookmarks
\usepackage{hypcap} % fix the links
\evensidemargin\oddsidemargin
\usepackage{graphicx}
\pagestyle{plain}
\usepackage{xcolor}
\newcommand\ToDo[1]{\textcolor{red}{#1}}
%\bibliographystyle{plainnat}
\usepackage{siunitx}
\usepackage{color}

\usepackage[draft,nomargin,inline]{fixme}
\fxsetface{inline}{\itshape}
\fxsetface{env}{\itshape}
%\fxuselayouts{margin}
%\fxuselayouts{inline}
\fxusetheme{color}

\usepackage{url}
\urldef{\mailsa}\path|{djukanovic, raidl}@ac.tuwien.ac.at,|
\urldef{\mailsb}\path|christian.blum@iiia.csic.es|
\newcommand{\keywords}[1]{\par\aDSvspace\baselineskip
	\noindent\keywordname\enspace\ignorespaces#1}

\usepackage{tikz}
\usetikzlibrary{positioning}
\definecolor{canaryyellow}{rgb}{1.0, 0.94, 0.0}
\definecolor{brightgreen}{rgb}{0.4, 1.0, 0.0}
\definecolor{jazzberryjam}{rgb}{0.65, 0.04, 0.37}

%defining of command

\newcommand\floor[1]{\lfloor#1\rfloor}
\newcommand\ceil[1]{\lceil#1\rceil}
\newcommand\str[1]{\texttt{#1}}
\newcommand\pL[1][]{\ensuremath{p^{\mathrm{L}#1}}}
\newcommand\pR[1][]{\ensuremath{p^{\mathrm{R}#1}}}
\newcommand\qL{\ensuremath{q^\mathrm{L}}}
\newcommand\qR{\ensuremath{q^\mathrm{R}}}
\newcommand\pLH{\ensuremath{\hat{p}^\mathrm{L}}}
\newcommand\pRH{\ensuremath{\hat{p}^\mathrm{R}}}
\newcommand{\Vext}{\ensuremath{V_\mathrm{{ext}}}}
\newcommand\UB{\ensuremath{\mathrm{UB}}}
\newcommand\Sigmand{\ensuremath{\Sigma^\mathrm{nd}}}
\renewcommand{\labelenumii}{\theenumii}
\renewcommand{\theenumii}{\theenumi.\arabic{enumii}.}
\setlength{\leftmarginii}{1.8ex}
\raggedbottom
\algnewcommand\algorithmicforeach{\textbf{for each}}
\algdef{S}[FOR]{ForEach}[1]{\algorithmicforeach\ #1\ \algorithmicdo}

% scaling factor for tables
\newcommand\tabscale{0.8}

\begin{document}
	
	%\setlength{\parindent}{0pt}  % disallow indentations
	%\numberwithin{table}{1}
	%\mainmatter  % start of an individual contribution
	
	% first the title is needed
	\title{Can greedy-like heuristics be useful for solving the Weighted Orthogonal Art Gallery Problem under regular grid discretization?}
	
	%
	\author[1]{Milan Predojevi\'c}
\author[1]{Marko Djukanovi\'c}
\author[1]{Milana Grbi\'c}
\author[1]{Dragan Mati\'c}
    \address[1]{Faculty of Natural Science and Mathematics, University of Banja Luka, Bosnia and Herzegovina}

	\begin{abstract}
		In this paper we deal with the Weighted Orthogonal Art Gallery Problem under the regular grid discretization. We propose a novel greedy approach which is based on balancing the trade off between the total sum of guards' costs and the total number of not yet covered points from the discretization. This new approach and an existing greedy algorithm are further  hybridized with the Integer Linear programming (ILP), originally formulated for the well known Minimum Set Cover problem. Experimental results show that the proposed greedy methods can achieve most of the optimal solution for a class of large area polygons, while for small area polygons, they achieve solutions of reasonable quality within lower runtime than the exact algorithms. 
	\end{abstract}
	\maketitle
	
	
	\section{Introduction}\label{sec:introduction}
	
	For a given polygon $P$, the \emph{Art Gallery Problem} (AGP) asks for a set of points $G$ of minimal cardinality,  such that for each point $y \in P$ there is $x \in G$ such that $xy \subset P$. We say that the point $y$ is covered by the point $x$, or $y$ is visible from $x$. Set $G$ is called \emph{guard set} of $P$ and the points from $G$ as \emph{guards}. In the \emph{Orthogonal Art Gallery Problem} (OAGP) we suppose that edges of the polygon are only horizontal and vertical w.r.t. the axes, i.e.  the angles allowed between adjacent edges are  $90^{\circ}$ or $270^{\circ}$. The original AGP was initially stated by Victor  Klee in 1973.~\cite{o1987art}.  The problem is motivated from installing the cameras inside a building (or gallery) such that the whole area of the building is covered. Orthogonality constraint naturally comes out from the orthogonality of the walls in buildings. Kahn et al.~\cite{kahn1983traditional} formulated and proofed that 	$\lfloor \frac{n}{4} \rfloor$ guards are  sufficient to cover an orthogonal polygon with $n$ vertices.      In this study, we are interested in the variant of the OAGP which allows only that guards are positioned at the vertices of polygon $P$. This restricted problem is known to be $\mathcal{NP}$--hard~\cite{schuchardt1995two,katz2008guarding}.  When it comes to the real situations (like installing the cameras in a building), it is justified to assume that the prices of cameras are not equal and may depends on several factors, like the quality of a camera   (respecting its range of spectrum of view)  or installation price at some specific parts of the building (like corners or tight places).  In the \emph{Weighted Orthogonal Art Gallery Problem} (WOAGP) the task is to place guards on some vertices of the orthogonal polygon which cover all points from $P$, such that the total sum of prices assigned to the chosen vertices is minimal. In this paper we consider WOAGP problem under the regular grid discretization of $P$, described in Subsection~\ref{sec:regulardiscretization}.
	
	It is well known that AGP can be reduced to \emph{the Minimum Set Cover Problem} (MSCP) by a discretization of the set of all points of the polygon $P$. The appropriate discretization should be performed in such a way that if each point from the discretized set $D(P)$ is covered, then the whole polygon $P$ is covered. After the discretization is made, for each vertex of the polygon $P$, a set of visible points from $D(P)$ is determined. In that way, the problem of determining the minimum number of guards covering the entire polygon is reduced to determining the minimum number of subsets of points, such that each point from $D(P)$ is included in at least one of the chosen subsets, which is MSCP. Analogously, WOAGP can be reduced to the \emph{Weighted Minimum Set Cover Problem} (WMSCP).
	
	Concerning the exact and heuristic techniques to solve OAGP, Couto et al.~\cite{couto2007exact} presented an exact and efficient exact algorithm for the OAGP based on preprocessing and refinement phases of the the discretized instance. In ~\cite{ghosh2010approximation} an approximate solution of the minimum vertex guard problem, which can be computed in $O(n^4)$ time and the size of the solution is at most $O(\log n)$ times the optimal. After that, on these constructed sets Johnson’s approximation algorithm ~\cite{johnson1974approximation} for the MSCP is applied. An anytime algorithm to compute successively better approximations of the optimum to Minimum Vertex Guard is proposed in ~\cite{tomas2003approximation}.  A major idea of this approach is exploring dominance of visibility regions to first detect pieces that are
	more difficult to guard. The same problem is solved   in ~\cite{tomas2006visibility} by applying successive approximations from  ~\cite{tomas2003approximation}.
	Tozoni et a. ~\cite{tozoni2013practical,tozoni2016algorithm}  presented an exact \emph{Integer Linear Programming}  (ILP)-based  algorithm, which iteratively generates upper and lower bounds through the resolution of discretized space of the AGP. Although many variants AGP are present in literature, WOAGP has not been so intensively studied, which motivated us to consider this problem.
	A comprehensive analysis of various greedy-like heuristics for the WMSCP was presented in \cite{vasko2016best}.
	More detailed overview of the extensive literature regarding SCP and AGP is out of the scope of this paper and for further reading we suggest review papers ~\cite{caprara2000algorithms,ren2010new,ghosh2010approximation2}.

	
	\subsection{Main contributions}
	The main contributions of this paper are:
	\begin{itemize}
		\item We developed a novel greedy approach which is based on balancing the trade off between the total sum of guards' costs and the total number of not yet covered points from the discretization.
	
		\item The  greedy algorithm from \cite{chvatal1979greedy} and the novel greedy algorithm are hybridized with the ILP.
		\item We considered different types of weights for our benchmarks, based on an approximation of the costs in real situations.
		\item In a comprehensive computational experiment,  we tested, analysed and checked the efficiency of the developed algorithms. The methods are then compared to the exact approaches ILP and \emph{Constraint Programming} (CP) w.r.t. the quality of obtained heuristic solutions as well as runtimes.
	\end{itemize}

	
	\subsection{The Regular Grid Discretization of Polygon}\label{sec:regulardiscretization}
	The discretization $D(P)$ of the polygon $P$ is built as follows. Consider the regular grid with resolution $\bigtriangleup_{x}\times\bigtriangleup_{y}$ starting at the lower left corner of the bounding box of polygon $P$ which height and weight are defined as follow:

	\begin{equation}
         \bigtriangleup_{x}=\min\{ |u_{x}-v_{x}|\mid u_{x}\neq v_{x}\} \mbox{ and }
         \bigtriangleup_{y}=\min\{ |u_{y}-v_{y}|\mid u_{y}\neq v_{y}\},
 	\end{equation}
where $ (u_{x},u_{y}),(v_{x},v_{y})$ are adjacent vertices of polygon $P$.
	
	\begin{algorithm}[!t]
		\caption{Discretization $D(P)$ of polygon $P$}\label{alg:discret}
		\begin{algorithmic}[1]
			\State \textbf{Input:} The set of vertices $V$ of polygon $P$
			\State \textbf{Output:} The discretization $D(P)$ of polygon $P$
			\State $BB \gets bounding\_box(P);$
			\State $\bigtriangleup_{x},\bigtriangleup_{y} \gets resolution(P);$
			\State $D(BB) \gets regular\_grid(BB,\bigtriangleup_{x},\bigtriangleup_{y});$
			\State $D(P) \gets D(BB) \cap P;$
			\State $D(P) \gets D(P) \cup V;$
		\end{algorithmic}
	\end{algorithm}

	All intersection points of this regular grid with polygon $P$ and all vertices of $P$ are added into the discrete set $D(P)$. The pseudocode which presents the building steps of the discretizations $D(P)$ is given in Algorithm~\ref{alg:discret}.
	
	\fxnote{TODO: zasto uzimamo ovakav tip diskretizacije, objasniti zasto id pokrivenosti diskretizacije ne slijedu pokrivenost poligona -- dvije slike dodaj gdje optimalan izbor cvorova pokriva poligon, i gdje optimalni izbor cvorova ne pokriva poligon...}
	
	\section{Exact methods}
	In this section we present the exact ILP and \emph{Constraint Programming} (CP) models for solving WMSCP  under regular grid discretization, which are used  in the rest of the paper.
	\subsection{Integer linear programming model}
	Let us suppose we are given a polygon $P$ and the discretization $D(P)$ of $P$.  The task we consider is covering all points from $D(P)$ by some vertices $V=\{v_1,...,v_n\}$ of $P$ such that the sum of their weights is minimized.
	The problem is related to the known MWSC problem as follows.
	Family $\mathcal{F}$ of nonempty sets consists of the sets
	$S_i \in \mathcal{F}$ which include points $p \in D(P)$ that are visible from guard $v_i\in V$, i.e., $pv_i \subset P$.  Note that set $S_i$ includes points $p\in D(P)$ that might also be included by some other guard $v_j\in V$, $i \neq j$. For each set $S_i$, the cost $c(S_i) = w_i$ is assigned.  In this way, our starting task is equivalent of finding a minimum cardinalty covering $\mathcal{C}\subseteq\{S_1,...,S_n\}$ of the set of points $D(P)$, that is
	$$ \bigcup_{C \in \mathcal{C}} C = D(P),$$ such that $\sum_{S_i \in \mathcal{C}} c(S_i)$ is minimized.  The ILP  model for the WSCP is already known from the literature, where  here adapted to WOAGP  as follows:

	\begin{align}
	&\sum_{i=1}^n w_ix_i \longrightarrow \min \\
	&\mbox{s.t.} \\
	&\sum_{j\in V} a_{ij}x_j \geq 1\ (\forall p_i\in D(P)) \label{eq:const-3}\\
	& x_j \in \{0,1\}, j \in V,
	\end{align}
	where
	$a_{ij} = \begin{cases}
	1, p_i \in V(j), \\
	0, \mbox{otherwise}
	\end{cases}$
	and $x_i = \begin{cases}
	1, \mbox{ if } \mbox{ the point } p_i \in \mathcal{C},\\
	0, \mbox{otherwise},
	\end{cases}$ \\

	where $V(j)$ is the set of all points from $D(P)$ that are visible from $j$-th vertex of polygon $P$.
	Set $Z = \{j \in V\mid x_i=1\}$ represents a solution of the problem w.r.t. discretization $D(P)$ of polygon $P$.
	Constraint~(\ref{eq:const-3}) enforces that any point $p_i \in D(P)$ will be visible from at least one guard from $Z$.
	
	In order to solve this model, we apply a general purpose solver \textsc{Cplex}~\cite{lima2010ibm}.
	\subsection{Constraint programming model} An equivalent CP model was implemented and tested by IBM ILOG CP Optimizer~\cite{laborie2018ibm}. In this case,  Constraint~(\ref{eq:const-3}) is transformed into
	\begin{equation}
	\bigvee_{ j \in V } (a_{ij} \wedge x_j) = 1,
	\end{equation}
	whereas the other constraints and the objective function are the same like in the above ILP model. Note that CP appraoch works in a branch-and-bound manner employing a constraint propagation and variable domain filtering~\cite{rossi2006handbook}.
	\section{Algorithmic Approaches for WOAGP}
	\subsection{Greedy approaches for solving WOAGP}
 Greedy algorithms produce a solution of reasonable quality within a short interval of time and are, in essence, easy to implement. Efficiency of such  heuristic is related to a greedy criterion utilized to expand current (non-complete, i.e., partial) solution to complete one. Among all candidates (solution components for expansion, that is not-yet-considered guards) to extend current partial solution,  we choose one with the smallest greedy value and add it to the current solution until it becomes complete (i.e., it covers all  points from $D(P)$).
	
	A general pseudocode of Greedy heuristics is given in Algorithm~\ref{alg:greedy}.
	
	\begin{algorithm}[!t]
		\caption{Greedy Heuristic}\label{alg:greedy}
		\begin{algorithmic}[1]
			\State \textbf{Input:} an instance of a problem
			\State \textbf{Output:} A (feasible) non-expandable solution (or reporting that no feasible solution)
			\State $s^{ps} \gets ()$ \hspace{0.3cm}// partial solution set to empty solution
			\While{$\text{Extend}(s^{ps}) \neq \emptyset$}
			\State Select solution component $e \in  \text{Extend}(s^{ps})$ \hspace{0.3cm}//\,w.r.t.\  some criterion $g$
			\State Extend $s^{ps}$ by $e$
			\EndWhile
		\end{algorithmic}
	\end{algorithm}
	\subsubsection{An existing greedy method}
	
	%  \subsection{Greedy Criterion based on Price-per-Unit}
	Concerning the greedy heuristic for the literature for solving WMSC problem~\cite{chvatal1979greedy, lovasz1975ratio}, one of the most efficient greedy heuristic was based on the following criterion:
	\begin{align}\label{eq:Lovasz}
	g_1(s^{ps}, v_i) = \frac{w_{i}}{ f(s^{ps} \cup \{v_i\})  - f(s^{ps})},
	\end{align}
	where
\begin{equation}\label{eq:sp}
    f(s^{ps}) = \left|\bigcup_{i \in s^{ps}} S_i \right|.
    \end{equation}
	This heuristic  also ensures an approximation with $O(\log(n))$ approximation factor.
	
	\subsubsection{A Novel Greedy Heuristic}
	In this subsection we present a novel greedy criterion. First, we introduce a term  ``incorrect point''. For a point from $D(P)$ we say that it is \textit{incorrect} if it is not covered by any guard from the current partial solution $s^{ps}$. Let us denote by $incorrect_{total}(s^{sp})$ the total number of incorrect points from discretization $D(P)$ w.r.t. $s^{ps}$. Note that  $incorrect_{total}(s^{ps}) = |D(P)|-f(s^{ps})$, where $f(s^{ps})$ is defined in (\ref{eq:sp}) and $|D(P)|$ is the cardinality of discretization set . Let $w_{total}$ be the total sum of all weights among all vertices. The greedy function w.r.t. partial solution $s^{ps}$ and  guard $v$ is defined as follows.
	\begin{equation}\label{eq:greedyfun2}
    g_2(s^{ps}, v)  =    \frac{\sum_{i \in s^{ps} \cup \{v\}} w_i}{w_{total}}+ \frac{incorrect_{total}(s^{ps}\cup v)}{|D(P)|}.
	\end{equation}

	In the Equation  (\ref{eq:greedyfun2}) both terms are normalized on order to achieve better balance between the total sum of guards' costs and the total number of not yet covered points from the discretization. The algorithm does not  ultimately prefer any of criteria for choosing next vertex. Although it cannot be a rule, it is justified to suppose that the second term has a more influence on the choice of the next vertex in earlier phases of the algorithm execution. At the beginning, more points are uncovered and the algorithm  chooses such vertices which cover larger area of the polygon. At the end of algorithm, it could be expected that the first term plays more significant role, since the total number of uncovered points is small.
	
	
	Ties occurred in the search are broken by using  price-per-unit heuristic which is stated as follows.
	For each not yet considered vertex $v_i$, we denote the region of polygon $P$ that is visible from $v_i$ by $\emph{Surf}(v_j)$. As the next candidate to extend $s^{ps}$, we choose such a guard,  with the smallest ratio between the price and the visible surface area. More precisely, this  criterion is given as:
	\begin{align}
	g'(s^{ps}, v_i) = \frac{w_{i}}{|\emph{Surf}({v_i})|},
	\end{align}

	In our experimental studies, we found out that this heuristic does not perform well on its own, but it represents a reasonable tie-breaking mechanism which boosts quality of the aforementioned greedy heuristics.
	\subsubsection{Partial calculation of greedy functions}
	
	In order to enable fast calculation of greedy functions described in previous subsections, we noticed that it could be useful to allow fast updating of the number of uncovered vertices. Therefore, we introduced two useful structures and two auxiliary functions:
	\begin{itemize}
		%\item structure \texttt{map<Point,list<Point>> Visibility} -- for each point of $D(P)$, we provide the list of guards which are visible from that point;
		\item structure \texttt{map<Point,int> numberOfGuards} -- for each point from $D(P)$,  we keep the number of guards in solution which cover that point;
		\item structure \texttt{set<Point> CoveredPoints} -- set of points from $D(P)$ that are covered by the partial solution;
		\item function \texttt{updateCoveredPointsAdd (vertex v)} -- does the update of  structures \texttt{CoveredPoints} and \texttt{numberOfGuards} by considering all new points which are covered by vertex $v$, when $v$  is added to partial solution; pseudocode of the procedure is shown in Algorithm \ref{alg:updateCoveredPointsAdd}.
		\item function \texttt{updateCoveredPointsRemove(vertex v)} -- does the update of structures \texttt{CoveredPoints} and \texttt{numberOfGuards} by removing all points which are covered only by guard $v$ , when $v$ is removed from partial solution; pseudocode  is shown in Algorithm \ref{alg:updateCoveredPointsRemove}.
	\end{itemize}
 Although the time complexity of both functions is $O(|D(P)|)$, from Algorithms \ref{alg:updateCoveredPointsAdd} and \ref{alg:updateCoveredPointsRemove} it can be noticed that they only depends on the number of points covered by the current vertex, which is expected to be significantly lower than $|D(P)|$. These two functions allow fast calculation of the functions $f$ and  $incorrect_{total}$: a candidate vertex $v$ is temporary added to the partial solution, new status is checked and the vertex is then removed from the solution.
	\begin{algorithm}[!t]
          	\caption{Function \texttt{updateCoveredPointsAdd}}\label{alg:updateCoveredPointsAdd}
          	\begin{algorithmic}[1]
          		\State \textbf{Input:} vertex $v$, $s[v]$:  list of points covered by $v$
          		\State \textbf{Output:} updated structures: coveredPoints and numberOfGuards
          		\For{$p$ in $S[v]$}
          		\State coveredPoints.insert($p$)
          		\State numberOfGuards.find($p$).second++
          		\EndFor
          	\end{algorithmic}
          \end{algorithm}

          %\begin{comment}

            \begin{algorithm}[!t]
          	\caption{Function \texttt{updateCoveredPointsRemove}}\label{alg:updateCoveredPointsRemove}
          	\begin{algorithmic}[1]
          		\State \textbf{Input:} vertex $v$, $s[v]$:  list of points covered by $v$
          		\State \textbf{Output:} updated structures coveredPoints and numberOfGuards
          		\For{$p \in S[v]$}
          		\State numberOfGuards.find($p$).second-- --
                \If {numberOfGuards.find($p$).second$=0$}
                    \State coveredPoints.remove($p$)
                    \EndIf
          		\EndFor
          	\end{algorithmic}
          \end{algorithm}
	 % \end{comment}
\subsubsection{A Hybrid of the \textsc{Greedy + \textsc{Cplex}}}
	The performance of \textsc{Cplex}  sooner or later degrades w.r.t. instance size due to the complexity of the problem. On the other hand, in the later stage, there is an increased chance for Greedy to worsen the obtained greedy solution. So, it makes sense to combine  partial solutions generated with a Greedy procedure over a few iterations  and only then to make use the \textsc{Cplex} to do the completion of the partial solution. In details, our approach consists of the following steps:
	\begin{enumerate}
		\item Run a Greedy method up to $K$ iterations (parameter) to obtain a partial solution $s^{ps}$ (therefore, $|s^{ps}| = K$);
		\item Take solution $s^{ps}$ and make it complete by solving a corresponding submodel via \textsc{Cplex}:
		\begin{itemize}
			\item \textsc{Cplex} solves corresponding sub-model which is formed by adding constraints $x_{i} = 1$, for all $v_i \in s^{ps}$ into the existing WOAGP model;
			\item a complete solution $\bar {s^p}$ is obtained;
		\end{itemize}
		\item Return $f(\bar {s^p})$.
	\end{enumerate}
 
	\begin{comment}
	
	\noindent \textbf{Improvements of the above method.} The above method can serve as a basic iteration
	of a more advanced techniques like ILP-LNS or CMSA. In this case, methods for destructing the solutions
	has to be proposed.  Underlying idea could be:
	\begin{itemize}
	\item remove $N$ guards with the largest costs out of $C'$
	\item remove $N$ guards which have a higher amount of points from $D(P)$ covered by other guards, represented by the function
	\begin{align}
	ratio(i) = \frac{\sum_{v \in V\setminus{ \{i\}}, j \in V(i)} 1_{j \mbox{ is veasible from } v} }{|V(i)|}.
	\end{align}
	\end{itemize}
	\end{comment}
	
	\section{Computational Results}

 We used two different kind of benchmark sets:
     \begin{itemize}
     	 \item \emph{MinArea} instances:\fxnote{TODO: Milan: main characteristics, why they differ from others}
     	 \item \emph{FAT} instances: \fxnote{TODO: Milan: main characteristics, why they differ from others}
     \end{itemize}
     For each benchmark set -- \emph{MinArea} and \emph{FAT} instances -- one polygon for each $n \in\{8,10,...,200\}$ has been generated, which makes 97 instances per each benchmark set. In overall, we have 194 polygons. 
	We included two kind of weights into each of (194) instances:
	\begin{itemize}
		\item \emph{topologically-based related weights} ($W0$): For each vertex $v_i$ of polygon $P$ we denote by $l_i$ and $l_{i+1}$ the lengths of two edges that come out of the vertex $v_i$. Then, we set $w_i := \frac{l_i + l_{i+1}}{2}$. This way of assigning weights is augmented by the fact that if the the arithmetic length of both edges that comes out of vertex $i$ is longer, it is expected that a guard can see a larger pieces of polygon $P$. This implies that the spectrum of camera installed at $v_i$ has to be larger, which implies that it should be of a higher quality, i.e,. a higher price. In case that we have the second edge begin smaller, we consider arithmetic mean of the both lengths as the weight of the vertex.
		\item \emph{point-based related weights} ($W1$): For each vertex $j$, we consider the number of points in $D(P)$ that are visible from the given vertex $(|V(j)|)$. Based on this, we assign prices to each vertex on the following way:
		\begin{equation}
		w_i = n \cdot \frac{|V(i)|}{|D(P)|}, i=1,...,n.
		\end{equation}
	\end{itemize}
    
     \subsection{Settings and the choice of the Parameters}
      All variants of our algorithms were implemented in C++ with g++~7.4 and the experiments were conducted in single-threaded mode on a machine with an Intel Xeon E5-2640 processor with 2.40 GHz and a memory limit of 8GB. The maximum computation time of each of our algorithms was set to $5\ min$.

      After conducting preliminary results, for
        \textsc{Greedy + \textsc{Cplex}} we decided to set up $K = \lceil 0.05 \cdot n \rceil$.

     The benchmark sets and the executable file of our software for this project are provided at \href{link}{link}.
	\subsection{Results and Discussion}
        The following algorithms are included in our computation:
        \begin{itemize}
        	\item  exact approaches: CP and ILP approach.The later is, henceforth, called \textsc{Cplex};
            \item 4 heuristic approaches:
           \begin{itemize}
           	\item two pure greedy methods, guided by $g_1$ and $g_2$, labeled as \textsc{Greedy-1}, \textsc{Greedy-2}, respectively;
            %\item three \textsc{Greedy+Shaking} variants, henceforth labeled by \textsc{Greedy}-$i$+Shaking, $i=1,2,3$, where $i$ stands for heuristic $g_i$ involved in the method;
            \item two \textsc{Greedy+Cplex} variants, henceforth labeled by  \textsc{Greedy-1}+\textsc{Cplex} and 
            \textsc{Greedy-2}+\textsc{Cplex}. 
        \end{itemize}
    \end{itemize}
            
        
         Summarized numerical results for each of our 6 variants of our algorithm are displayed 
         in Tables~\ref{small_wo} --\ref{large_w1}. For each table, the average results are reported for each subset of instances grouped w.r.t. a specific kind of instances and a specific weight (97 instances per each group). Each of the tables list the considered algorithms in the first column. Starting with column two,  each of the lines provide detailed statistics for respective algorithm.  The statistics report: the average solution quality ($\overline{|s|}$), the average time ($\overline{t}[s]$), the average number of guards which are included in each of the solutions ( $\overline{|g|}$), the number of solutions which match to the optimum solutions (\#hitsMinH), standard deviation of the obtained solutions w.r.t. optimal solutions ($\sigma(Opt)$), then the number of minimal solutions w.r.t. the four heuristics achieved by the respective algorithm (\#hitsMinH), standard deviation w.r.t. best solutions achieved by the four heuristic algorithms   ($\sigma(MinH)$).  An asterisk in front of a number means that the exact approach  could prove optimality in all cases.      
         
    
     \begin{table}[]
     	\centering
     	 \scalebox{0.83}{
     	\begin{tabular}{l|lllllll}    \hline
     		Alg.           & $\overline{|s|}$ & $\overline{t}[s]$ &   $\overline{|g|}$ &optHit & $\sigma(Opt)$ & \#hitsMinH & $\sigma(MinH)$ \\ \hline
     		\textsc{Cplex}            & *15.93  & 0.152    & 18.93     & 97/97     & 0.00    & N/A        & N/A      \\
     		\textsc{Cp}               & 16.04  & 98.215   & 18.96     & 67/97     & 0.02    & N/A        & N/A      \\ \hline
     		\textsc{Greedy-1}             & 22.63  & \textbf{0.002}    & 26.28     & 0/97      & 0.79    & 3/97       & 7.61     \\
     		\textsc{Greedy-2}          & 19.97  & \textbf{0.002}    & 20.14     & 0/97      & 0.44    & 0/97       & 0.20     \\
     		   \textsc{Greedy-1+Cplex}  & 18.49  & 0.073    & 21.19     & \textbf{2/97}      & 0.30    & \textbf{57/97}      & \textbf{0.06}     \\
     		   \textsc{Greedy-2+Cplex} & \textbf{18.39}  & 0.004    & \textbf{19.59}     & 0/97      & \textbf{0.26}    & 53/97      & 0.04  \\ \hline 
     	\end{tabular}}
        \caption{The results on the benchmark set $MinArea$ for the type of weight $W0$.}\label{small_wo}
     \end{table}
 
      \begin{table}[]
      	\centering
      	\scalebox{0.83}{
      	\begin{tabular}{l|lllllll} \hline
      		Alg.           & $\overline{|s|}$ & $\overline{t}[s]$ &  $ \overline{|g|}$ &optHit & $\sigma(Opt)$ & \#hitsMinH & $\sigma(MinH)$ \\ \hline
      		\textsc{Cplex}            & *36.03  & 0.003   & 18.93     & 97/97     & 0.00    & N/A        & N/A      \\
      		\textsc{Cp}               & *36.03  & 18.953   & 18.93     & 97/97     & 0.00    & N/A        & N/A      \\ \hline
      		\textsc{Greedy-1}             & 39.28  & \textbf{0.001}    & 20.15     & 2/97      & 0.35    & 4/97       & 0.24     \\
      		\textsc{Greedy-2}          & 40.92  & \textbf{0.001}    & 20.96     & 0/97      & 0.51    & 0/97       & 0.40     \\
      		   \textsc{Greedy-1+Cplex}  & 37.78  & 0.006    & 19.41     & \textbf{4/97}      & 0.21    & 19/97      & \textbf{0.08}     \\
      		   \textsc{Greedy-2+Cplex} & \textbf{37.31}  & 0.003    & \textbf{19.58}     & 2/97      & \textbf{0.15}    & \textbf{81/97}     & {0.19}   \\ \hline  
      	\end{tabular}}
         \caption{The results on the bemchmark set $MinArea$ for the  type of weight $W1$.}\label{small_w1}
      \end{table}
  
     \begin{table}[]
      \centering
       \scalebox{0.85}{
     	\begin{tabular}{l|lllllll} \hline
     		Alg.           & $\overline{|s|}$ & $\overline{t}[s]$ &   $\overline{|g|}$ &optHit & $\sigma(Opt)$ & \#hitsMinH & $\sigma(MinH)$ \\ \hline
     		\textsc{Cplex}            & *2.62   & 0.02    & 7.32      & 97/97     & 0.00    & N/A        & N/A      \\
     		\textsc{Cp}               & *2.62   & 8.17    & 7.32      & 97/97     & 0.00    & N/A        & N/A      \\ \hline
     		\textsc{Greedy-1}             & 3.39   & 0.35    & 7.81      & 0/97      & 0.09    & 23/97      & 0.05     \\
     		\textsc{Greedy-2}          & 3.52   & \textbf{0.32}    & 6.61      & 2/97      & 0.10    & \textbf{28/97}      & 0.08     \\
     		   \textsc{Greedy-1+Cplex}  & 3.34   & 0.35    & 8.91      & 4/97      & \textbf{0.08}    & 25/97      & \textbf{0.04}     \\
     		   \textsc{Greedy-2+Cplex} & \textbf{3.32}   & 0.33    & \textbf{6.81}      & \textbf{6/97}      & \textbf{0.08}    & 25/97      & \textbf{0.04}  \\ \hline   
     	\end{tabular}}
          \caption{The results on the benchmark set $FAT$ for the type of weight $W0$.} \label{large_w0}
     \end{table}
 
    \begin{table}[]
    	\centering
    	\scalebox{0.85}{
 	  \begin{tabular}{l|lllllll} \hline
 		Alg.           & $\overline{|s|}$ & $\overline{t}[s]$ &   $\overline{|g|}$ & optHit & $\sigma(Opt)$ & \#hitsMinH & $\sigma(MinH)$ \\ \hline
 		\textsc{Cplex}            & *6.72   & 0.02    & 6.44      & 97/97     & 0.00    & N/A        & N/A      \\
 		\textsc{Cp}               & *6.72   & 0.36    & 6.44      & 97/97     & 0.00    & N/A        & N/A      \\ \hline
 		\textsc{Greedy-1}             & 6.77   & \textbf{0.31}    & 6.49      & 92/97     & 0.02    & 92/97      & 0.04     \\
 		\textsc{Greedy-2}          & 6.77   & \textbf{0.31}    & 6.49      & 92/97     & 0.02    & 92/97      & 0.04     \\
 		\textsc{Greedy-1+Cplex}  & \textbf{6.73}   & 0.32    & \textbf{6.45}      & \textbf{96/97}     & \textbf{0.01}    & \textbf{96/97}      & \textbf{0.03}     \\
 		\textsc{Greedy-2+Cplex} & 7.42   & \textbf{0.31}    & 6.46      & 90/97     & 0.33    & 92/97      & 0.33    \\ \hline
 	\end{tabular}}
    \caption{The results on the benchmark set $FAT$ for the type of weight $W1$.}\label{large_w1}
   \end{table}
  
        From the numerical results, we observe the following conclusions concerning exact solving:
        \begin{itemize}
        	\item  \textsc{Cplex} could solve all solution to prove optimality within a fraction of a second;
        	\item   \textsc{Cp} was able to solve all those instances with weight $W1$ but with signifficanlty more time than the time required from \textsc{Cplex};
        	\item For the subset of instances $MinArea$ which includes weight $W0$, the performance of Cp degrades signifficantlly, which can be seen  by the respective avg. runtime and the number of proven optimal solutions found (67/97).
        \end{itemize}
     From the numerical results, we observe the following conclusions concerning heuristic solving:
     \begin{itemize}
     	\item  In case of the instances that include small-area polygons ($Minarea$) and weight $W0$ (Table~(\ref{small_wo})): A novel \textsc{Greedy-2} was able to significantly outperform the \textsc{Greedy-1} from literature. In this case, the best performing heuristic algorithms is hybrid \textsc{Greedy-2+Cplex} which needs two times lower avg. runtime to finish than the pure greedy methods. The avg. runtimes of all four greedy methods is an order of magnitude lower than the avg. runtime of \textsc{Cplex}. Pure greedy methods deliver solutions of reasonable quality within 20\% of the quality of optimal solutions but needs time which is 7-8 times lower then the avg. runtime of \textsc{Cplex}. 
     	\item  In case of the instances that include small-area polygons and weight $W1$ (Table~(\ref{small_w1}): the best heuristic appraoch is \textsc{Greedy-2+Cplex} which is able to deliver solutions which are within 3\% of the optimum ones. Note that in 81 instances (out of 91) \textsc{Greedy-2+Cplex} is able to deliver equally good or better results that the other heuristic approaches. 
     	The results of pure greedy method \textsc{Greedy-1} outperform the results of \textsc{Greedy-2}. These two methods are effective since they deliver solutions which are within 10\% of the \textsc{Cplex} results but need 3 times less time. Te result of \textsc{Greedy-1} match just in two cases to the optimum.
     	
     	\item In case of the instances that include large-area polygons ($FAT$) and weight $W0$ (Table~(\ref{large_w0}): exact approach \textsc{Cplex} is a clear winner concerning avg solutions' quality as well as the avg. runtimes;  the best heuristic algorithm w.r.t. solution quality is \textsc{Greedy-2 +Cplex}. It also match in the most cases (on 6 instances) to the optimum results. Its avg. results are within 22\% of the avg. of  optimal solutions.  The pure \textsc{Greedy-1} slightly outperforms the pure greedy \textsc{Greedy-2}. It is interesting that \textsc{Greedy-1} chooses in average a  higher number of guards in comparison to the avg. number of guards of \textsc{Greedy-2}. The runtimes of our heuristic approaches are an order of magnitude lower than the time of \textsc{Cplex} approach. 
     	\item  In case of the instances that include large-area polygons ($FAT$) and weight $W1$ (Table~(\ref{large_w1}): the best heuristic algorithms w.r.t. solution quality is \textsc{Greedy-1+Cplex} which is able to match in 96 instances the quality of optimal solutions. Slightly worse results are \textsc{Greedy-1} and \textsc{Greedy-2}. The obtained (heuristic) solutions of the two approaches are within 1\% of optimal solutions and they are
     	able to reach the quality of the optimal solution in 92 instances. Unfortunately, the runtimes in comparison to the runtimes of \textsc{Cplex} are significantly higher for all of our heuristic approaches. However, the avg. runtime for \textsc{Cp} is a bit higher that the runtime of the heuristic approaches. 
     	\item \fxnote{TODO: pokrivenost...}
     \end{itemize}
    
	\section{Conclusions and Future Work}
	
	
	
	\bibliographystyle{abbrv}
	\bibliography{bib}
	
	
\end{document}
